%---------------------------------------------------------------------
%
%                      resumen.tex
%
%---------------------------------------------------------------------
%
% Contiene el cap�tulo del resumen.
%
% Se crea como un cap�tulo sin numeraci�n.
%
%---------------------------------------------------------------------

\chapter{Abstract}
\cabeceraEspecial{Resumen}

%En este proyecto, en primer lugar, se implementar� la Unidad Central de Procesamiento (CPU) la cual ser� capaz de ejecutar un subconjunto de instrucciones del repertorio THUMB-2 propio de algunas arquitecturas ARM. 
In the first place a Central Processing Unit (CPU) will be designed and implemented. This CPU will be able to execute a sub-set of the THUMB-2 instruction set from the ARM architectures. 

%Los sistemas electr�nicos, y los procesadores en particular, son susceptibles de sufrir fallos que interfieren en los procesos que lleven a cabo modificando los datos internos, y en consecuencia se produzcan resultados no deseados. Para evitar que esto suceda se aplicar�n t�cnicas de tolerancia a fallos sobre la CPU.  
Electronic systems, and microprocessors in particular, are susceptible to faults that interfere in the normal execution of the CPU, this faults modify the internal data and produces unwanted results. To avoid this happens, fault tolerant techniques will be applied on the CPU.


%Para finalizar, se comprobar� el funcionamiento del procesador antes y despu�s de a�adir los mecanismos de tolerancia a fallos, y en �ste ultimo caso se validar� que se consigue aumentar el grado de fiabilidad de la CPU.
Finally, the correct operation of the microprocessor will be tested before and after applying the fault tolerance techniques, for the latter case the fault sensitivity will be tested for an increase in the fiability of the system.

\vspace{5em}
\textbf{Keywords}:\,\relax%
Microprocessor, ARM, THUMB2, Fault Tolerance, TMR, Reliability

\endinput


%---------------------------------------------------------------------
%
%                          Cap�tulo 8
%
%---------------------------------------------------------------------

\chapter{Conclusions} % y futuras l�neas de trabajo}
\label{conclusions}

%\begin{FraseCelebre}
%\begin{Frase}
%...
%\end{Frase}
%\begin{Fuente}
%...
%\end{Fuente}
%\end{FraseCelebre}

\section{Conclusions}
\label{conclusions:conclusions}


%Se ha dise�ado una Unidad Central de Proceso (CPU) segmentada en 5 etapas, capaz de ejecutar un conjunto reducido de instrucciones conocido como THUMB-2, concretamente es un subconjunto de instrucciones de la arquitectura ARM. Esta CPU ejecuta correctamente instrucciones aritm�tico-l�gicas, instrucciones de acceso a memoria e instrucciones de bifurcaci�n o salto.
A Central Processing Unit (CPU) has been developed. This CPU has a 5 stage pipeline, and it is capable of executing a reduce instruction set known as THUMB-2, which is a sub-set of ARM architecture instruction set. This CPU  executes arithmetic and logic instructions, memory access instructions and branch instructions.

%Una vez obtenido el procesador con un repertorio de instrucciones capaz de ejecutar programas sencillos, se ha aumentado el grado de fiabilidad de la CPU dise�ada aplicando la t�cnica de tolerancia a fallos "`Triple Modular Redundancy"'. Se han triplicado los registros que almacenan informaci�n entre las etapas de la CPU segmentada e insertado votadores de mayor�a para obtener el valor correcto.
In a second implementation, the CPU's reliability level was enhanced by applying fault tolerance techniques, specifically the "`Triple Modular Redundancy"' (TMR) technique. The pipeline registers have been tripled and majority voters have been inserted to obtain the correct values.

%Para finalizar se ha demostrado, a base de simulaciones sobre el procesador dise�ado en este proyecto, que al aplicar t�cnicas como la TMR en la CPU se han conseguido tolerar fallos que habr�an alterado el funcionamiento del programa. El mecanismo de tolerancia aplicado ha permitido tolerar un fallo en cada elemento triplicado.
Finally, the simulations performed on the CPU developed had shown that applying fault tolerance techniques, as TMR, is possible to tolerate faults that would cause malfunction on the system. The fault tolerance mechanism applied allows a correct function of the system with only one fault in each tripled element.

These system properties are the project goals. All of them have been accomplished.
 
The CPU executes THUMB2 instructions, therefore it runs code generated by ARM compilers. Because of its simplicity, it can be integrated in  embedded systems with few resources. In this work, it have been shown that the TMR is a technique to produce reliable CPUs.
 

 

%\section{Futuras l�neas de trabajo}
%\label{conclusions:futuras}

%Las posibilidades futuras lineas de trabajo para continuar con este desarrollo de este proyecto podr�an incluir ampliaciones en la funcionalidad del sistema o aumentar la tolerancia a fallos. 
%
%Si se desea ampliar la funcionalidad se podr� conseguir aumentando el repertorio de instrucciones. En nuestro caso solo se han implementado instrucciones que permiten ejecutar programas b�sicos. Igualmente se pueden modificar las memorias de datos e instrucciones para permitir el uso de memorias externas e insertando memorias cach�. O bien se puede integrar la CPU junto a controladores de diferentes tipos.
%
%Por otro lado, si se desea aumentar la tolerancia a fallos, se puede aplicar la t�cnica TMR sobre otros componentes de la misma forma que se han aplicado sobre los registros de control. Tambi�n se pueden aplicar nuevas t�cnicas para tolerar otros tipos de fallos, tal es el caso de aplicar la re-configuraci�n est�tica o din�mica para que el sistema sea capaz de tolerar fallos permanentes adem�s de los fallos transitorios.







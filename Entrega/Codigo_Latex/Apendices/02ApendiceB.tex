%---------------------------------------------------------------------
%
%                          Ap�ndice B
%
%---------------------------------------------------------------------

\chapter{C�digo de simulaci�n con fallos}
\label{apendiceB}

Este c�digo se compone de 4 partes:

\begin{itemize}
	\item \textbf{Declaraci�n de componentes y se�ales:} Se declaran los componentes y las se�ales necesarias para controlar la simulaci�n. Esto incluye las se�ales de entrada y salida del procesador, junto a las se�ales esp�a\footnote{Estas se�ales solo se pueden utilizar con el software "`ModelSim"' de Altera.} que mostrar�n los valores de se�ales internas del sistema.
	\item \textbf{Proceso de conexi�n:} Se establece la conexi�n entre las se�ales esp�a y las se�ales internas del sistema.
	\item \textbf{Proceso de control:} Se realizan las comprobaciones para asegurar el correcto funcionamiento del procesador.
	\item \textbf{Introducci�n de fallos:} Se fuerzan se�ales internas del procesador a valores incorrectos.
\end{itemize}

\newpage

%-------------------------------------------------------------------
\section{Testbench para ejecuci�n con inserci�n de fallos en CPU est�ndar}
%-------------------------------------------------------------------
\label{apendiceB:tbcontrol}

\lstset{language=VHDL, breaklines=true, basicstyle=\footnotesize, numbers=left}
\begin{lstlisting}[frame=single]
[...]
	
	-- Test de funcionamiento
	spy_proc: process
	begin		
	
[...]

		assert spy_PC = std_logic_vector(to_unsigned(0, 32)) report "ERROR: Inicio" severity ERROR;      
	wait for clk_period*5; -- Espera para que tengan efecto los cambios (Procesador de cinco ciclos)
--LDR R2, R0, #5
		assert spy_PC = std_logic_vector(to_unsigned(20, 32)) report "ERROR: PC 20" severity ERROR;
		assert spy_R2 = std_logic_vector(to_unsigned(5, 32)) report "ERROR: LDR R2, R0, #5" severity ERROR;
	wait for clk_period;
--MOV R1, #25
		assert spy_PC = std_logic_vector(to_unsigned(24, 32)) report "ERROR: PC 24" severity ERROR;
		assert spy_R1 = std_logic_vector(to_unsigned(25, 32)) report "ERROR: MOV R1, #25" severity ERROR;
	wait for clk_period;
--MOV R3, #0
		assert spy_PC = std_logic_vector(to_unsigned(28, 32)) report "ERROR: PC 28" severity ERROR;
		assert spy_R3 = std_logic_vector(to_unsigned(0, 32)) report "ERROR: MOV R3, #0" severity ERROR;
	wait for clk_period;
--MOV R5, #1
		assert spy_PC = std_logic_vector(to_unsigned(32, 32)) report "ERROR: PC 32" severity ERROR;
		assert spy_R5 = std_logic_vector(to_unsigned(1, 32)) report "ERROR: MOV R5, #1" severity ERROR;
	wait for clk_period;

[...]

--STR R3, R0, #0
	wait for clk_period*12;
		assert spy_PC = std_logic_vector(to_unsigned(96, 32)) report "ERROR: PC 296" severity ERROR;
		assert spy_M0 = std_logic_vector(to_unsigned(125, 32)) report "ERROR: STR R3, R0, #0" severity ERROR;
	wait;
	end process;
[...]
\end{lstlisting}

\newpage 

\lstset{language=VHDL, breaklines=true, basicstyle=\footnotesize, numbers=left, firstnumber=last}
\begin{lstlisting}[frame=single]
[...]

   -- Inserci�n de fallos
   fault_proc: process
   begin		
      -- Fallo 1: Forzar Data read de memoria (bit 0) a 0 (Data read != 5 => Data read = 4)
      signal_force("/TB_ejecucion_fallos/uut/MEM_out_MEMbus_reg(0)", "0", 140 ns, freeze, 145 ns, 1); -- f_MEMbus

      -- Fallo 2: Forzar Data read de memoria (bit 0) a 0 (Data read != 5 => Data read = 4)
      signal_force("/TB_ejecucion_fallos/uut/EXE_out_MEM_control_reg(0)", "1", 270 ns, freeze, 275 ns, 1);

      -- Fallo 3: Forzar Data read de memoria (bit 0) a 0 (Data read != 5 => Data read = 4)
      signal_force("/TB_ejecucion_fallos/uut/ID_out_busB_reg(0)", "1",  530 ns, freeze, 535 ns, 1);
      
    wait;
   end process;
   
END;
\end{lstlisting}




\newpage

%-------------------------------------------------------------------
\section{Testbench para ejecuci�n con inserci�n de fallos en CPU tolerante a fallos}
%-------------------------------------------------------------------
\label{apendiceB:tbfallos}


\lstset{language=VHDL, breaklines=true, basicstyle=\footnotesize, numbers=left}
\begin{lstlisting}[frame=single]
[...]
	
	-- Test de funcionamiento
	spy_proc: process
	begin		
	
[...]

		assert spy_PC = std_logic_vector(to_unsigned(0, 32)) report "ERROR: Inicio" severity ERROR;      
	wait for clk_period*5; -- Espera para que tengan efecto los cambios (Procesador de cinco ciclos)
--LDR R2, R0, #5
		assert spy_PC = std_logic_vector(to_unsigned(20, 32)) report "ERROR: PC 20" severity ERROR;
		assert spy_R2 = std_logic_vector(to_unsigned(5, 32)) report "ERROR: LDR R2, R0, #5" severity ERROR;
	wait for clk_period;
--MOV R1, #25
		assert spy_PC = std_logic_vector(to_unsigned(24, 32)) report "ERROR: PC 24" severity ERROR;
		assert spy_R1 = std_logic_vector(to_unsigned(25, 32)) report "ERROR: MOV R1, #25" severity ERROR;
	wait for clk_period;
--MOV R3, #0
		assert spy_PC = std_logic_vector(to_unsigned(28, 32)) report "ERROR: PC 28" severity ERROR;
		assert spy_R3 = std_logic_vector(to_unsigned(0, 32)) report "ERROR: MOV R3, #0" severity ERROR;
	wait for clk_period;
--MOV R5, #1
		assert spy_PC = std_logic_vector(to_unsigned(32, 32)) report "ERROR: PC 32" severity ERROR;
		assert spy_R5 = std_logic_vector(to_unsigned(1, 32)) report "ERROR: MOV R5, #1" severity ERROR;
	wait for clk_period;

[...]

--STR R3, R0, #0
	wait for clk_period*12;
		assert spy_PC = std_logic_vector(to_unsigned(96, 32)) report "ERROR: PC 296" severity ERROR;
		assert spy_M0 = std_logic_vector(to_unsigned(125, 32)) report "ERROR: STR R3, R0, #0" severity ERROR;
	wait;
	end process;
[...]
\end{lstlisting}

\newpage 


\vspace{15mm}

\lstset{language=VHDL, breaklines=true, basicstyle=\footnotesize, numbers=left, firstnumber=last}
\begin{lstlisting}[frame=single]
[...]

	-- Inserci�n de fallos
	fault_proc: process
	begin		
		-- Fallo 1: Forzar Data read de memoria (bit 0) a 0 (Data read != 5 => Data read = 4)
		signal_force("/TB_ejecucion_fallos/uut/r_MEM_out_MEMbus/regs(0)(0)", "0", 140 ns, freeze, 145 ns, 1); -- f_MEMbus

		-- Fallo 2: Forzar Data read de memoria (bit 0) a 0 (Data read != 5 => Data read = 4)
		signal_force("/TB_ejecucion_fallos/uut/r_EXE_out_MEM_control/regs(0)(0)", "1", 270 ns, freeze, 275 ns, 1);

		-- Fallo 3: Forzar Data read de memoria (bit 0) a 0 (Data read != 5 => Data read = 4)
		signal_force("/TB_ejecucion_fallos/uut/r_ID_out_busB_reg/regs(0)(0)", "1",  530 ns, freeze, 535 ns, 1);   
	wait;
	end process;
   
END;
\end{lstlisting}





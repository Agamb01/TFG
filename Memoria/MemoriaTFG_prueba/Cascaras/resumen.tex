%---------------------------------------------------------------------
%
%                      resumen.tex
%
%---------------------------------------------------------------------
%
% Contiene el cap�tulo del resumen.
%
% Se crea como un cap�tulo sin numeraci�n.
%
%---------------------------------------------------------------------

\chapter{Resumen}
\cabeceraEspecial{Resumen}

Los sistemas electr�nicos son susceptibles a que part�culas de energ�a colisionen contra los componentes nanom�tricos que los componen, interfieran en el proceso que est�n llevando a cabo, modifiquen los datos internos, y en consecuencia se produzcan resultados no deseados. Esto incluye a los microprocesadores que se utilizan en los dispositivos de prop�sito general o espec�fico. 

En algunas �reas de aplicaci�n �stas part�culas se encuentran en una mayor cantidad y con una mayor energ�a. Cualquiera de ambas condiciones supone un aumente en la probabilidad de que ocurran colisiones y el sistema funcione de forma impredecible. 

Este proyecto busca implementar una Unidad de Central de Procesamiento (CPU) capaz de funcionar correctamente a�n recibiendo interferencias de estas part�culas. El proyecto se ha abordado en tres campos:

\begin{enumerate}
	\item En primer lugar se implementar� la Unidad Central de Procesamiento la cual, basada en la arquitectura DLX estudiada en los cursos de estructura y arquitectura de computadores, se adaptar� al repertorio de instrucciones elegido.
	\item A continuaci�n se aplicar�n t�cnicas de tolerancia a fallos sobre la CPU.
	\item Para finalizar, se comprobar� que la tolerancia a fallos se ha aplicado satisfactoriamente consigui�ndose aumentar el grado de fiabilidad del sistema.
\end{enumerate}

\endinput
% Variable local para emacs, para  que encuentre el fichero maestro de
% compilaci�n y funcionen mejor algunas teclas r�pidas de AucTeX
%%%
%%% Local Variables:
%%% mode: latex
%%% TeX-master: "../Tesis.tex"
%%% End:

%---------------------------------------------------------------------
%
%                      resumen.tex
%
%---------------------------------------------------------------------
%
% Contiene el cap�tulo del resumen.
%
% Se crea como un cap�tulo sin numeraci�n.
%
%---------------------------------------------------------------------

\chapter{Resumen}
\cabeceraEspecial{Resumen}


Este proyecto, en primer lugar se implementar� la Unidad Central de Procesamiento (CPU) la cual,  usando al repertorio de instrucciones ARM .... 32 bits ... .

Los sistemas electr�nicos, y los procesadores particular, son susceptibles a fallos que interfieran en el proceso que est�n llevando a cabo, modifiquen los datos internos, y en consecuencia se produzcan resultados no deseados.  

Vamos a re-dise�ar  CPU con  t�cnicas de tolerancia a fallos , para que sea capaz de funcionar correctamente ante un numero limitado fallos transitorios. 

	 Para finalizar, se comprobar� el funcionamiento del procador antes y despues de a�adir los mecanismos de tolerancia a fallos, y en este ultimo caso validar/verficar que se consigue aumentar el grado de fiabilidad de la cpu.
 

\endinput







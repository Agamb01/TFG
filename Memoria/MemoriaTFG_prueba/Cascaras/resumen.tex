%---------------------------------------------------------------------
%
%                      resumen.tex
%
%---------------------------------------------------------------------
%
% Contiene el cap�tulo del resumen.
%
% Se crea como un cap�tulo sin numeraci�n.
%
%---------------------------------------------------------------------

\chapter{Resumen}
\cabeceraEspecial{Resumen}

En este proyecto, en primer lugar, se implementar� la Unidad Central de Procesamiento (CPU) la cual ser� capaz de ejecutar un subconjunto de instrucciones del repertorio THUMB-2 propio de algunas arquitecturas ARM. 

Los sistemas electr�nicos, y los procesadores en particular, son susceptibles de sufrir fallos que interfieren en los procesos que lleven a cabo modificando los datos internos, y en consecuencia se produzcan resultados no deseados. Para evitar que esto suceda se aplicar�n t�cnicas de tolerancia a fallos sobre la CPU.  

Para finalizar, se comprobar� el funcionamiento del procesador antes y despu�s de a�adir los mecanismos de tolerancia a fallos, y en �ste ultimo caso se validar� que se consigue aumentar el grado de fiabilidad de la CPU.

\endinput







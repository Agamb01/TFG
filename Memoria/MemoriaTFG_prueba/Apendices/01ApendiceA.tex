%---------------------------------------------------------------------
%
%                          Ap�ndice 1
%
%---------------------------------------------------------------------

\chapter{C�digo de simulaci�n sin fallos}
\label{apendiceA}

Este c�digo se compone de 3 partes:

\begin{itemize}
	\item \textbf{Declaraci�n de componentes y se�ales:} Se declaran los componentes y las se�ales necesarias para controlar la simulaci�n. Esto incluye las se�ales de entrada y salida del procesador, junto a las se�ales esp�a\footnote{Estas se�ales solo se pueden utilizar con el software "`ModelSim"' de Altera.} que mostrar�n los valores de se�ales internas del sistema.
	\item \textbf{Proceso de conexi�n:} Se establece la conexi�n entre las se�ales esp�a y las se�ales internas del sistema.
	\item \textbf{Proceso de control:} Se realizan las comprobaciones para asegurar el correcto funcionamiento del procesador.
\end{itemize}


\newpage 

%-------------------------------------------------------------------
\section{Testbench para ejecuci�n de control}
%-------------------------------------------------------------------
\label{apendiceA:tbcontrol}

\lstset{language=VHDL, breaklines=true, basicstyle=\footnotesize, numbers=left}
\begin{lstlisting}[frame=single]
----------------------------------------------------------------
-- Company: Universidad Complutense de Madrid
-- Engineer: Andres Gamboa Melendez
-- 
-- Module Name: TB_ejecucion_normal - Testbench 
-- Project Name: ARM compatible micro-processor
-- Target Devices: Nexys4
-- Tool versions: Xilinx ISE Webpack 14.4
-- Description: Prueba de control para la ejecucion del programa 
--              multiplicacion basado en bucle de sumas.
-- 
-- VHDL Test Bench Created by ISE for module: cpu
-- 
----------------------------------------------------------------
library modelsim_lib;
use modelsim_lib.util.all;

LIBRARY ieee;
USE ieee.std_logic_1164.ALL;
USE ieee.numeric_std.ALL;
 
ENTITY TB_ejecucion_normal IS
END TB_ejecucion_normal;
 
ARCHITECTURE behavior OF TB_ejecucion_normal IS 

	-- Component Declaration for the Unit Under Test (UUT) 
	COMPONENT cpu
	PORT(
		clk : IN  std_logic;
		rst : IN  std_logic;
		led : OUT  std_logic_vector(15 downto 0)
	);
	END COMPONENT;

	--Inputs
	signal clk : std_logic := '0';
	signal rst : std_logic := '0';

	--Outputs
	signal led : std_logic_vector(15 downto 0);

	--Spy signals
	signal spy_R0    : std_logic_vector(31 downto 0);
	signal spy_R1    : std_logic_vector(31 downto 0);
	signal spy_R2    : std_logic_vector(31 downto 0);
	signal spy_R3    : std_logic_vector(31 downto 0);
	signal spy_R4    : std_logic_vector(31 downto 0);
	signal spy_R5    : std_logic_vector(31 downto 0);
	signal spy_M0    : std_logic_vector(31 downto 0);
	signal spy_M5    : std_logic_vector(31 downto 0);
\end{lstlisting}

\newpage

	\lstset{language=VHDL, breaklines=true, basicstyle=\footnotesize, numbers=left, firstnumber=last}
\begin{lstlisting}[frame=single]
	signal spy_PC    : std_logic_vector(31 downto 0);
	signal spy_INSTR : std_logic_vector(31 downto 0);

	-- Clock period definitions
	constant clk_period : time := 10 ns;
	signal clk_cycle : integer := 0;

BEGIN
 
	-- Instantiate the Unit Under Test (UUT)
	uut: cpu PORT MAP (
		clk => clk,
		rst => rst,
		led => led
	);

	-- Clock process definitions
	clk_process :process
	begin
		clk <= '0';
		wait for clk_period/2;
		clk <= '1';
		wait for clk_period/2;
	end process;
 
	-- Cycle process definitions
	cycle_process: process
	begin
		wait for 100 ns;
		wait for clk_period/2;
		clk_cycle <= clk_cycle + 1;
		loop
			wait for clk_period*2;
			clk_cycle <= clk_cycle + 2;
		end loop;
	end process;
	
	-- Configurar se�ales esp�as
	spy_init : process
	begin
		init_signal_spy("/TB_ejecucion_normal/uut/i_ID/i_pID/i_RegisterBank/regs(0)","/TB_ejecucion_normal/spy_R0", 1);
		init_signal_spy("/TB_ejecucion_normal/uut/i_ID/i_pID/i_RegisterBank/regs(1)","/TB_ejecucion_normal/spy_R1", 1);
		init_signal_spy("/TB_ejecucion_normal/uut/i_ID/i_pID/i_RegisterBank/regs(2)","/TB_ejecucion_normal/spy_R2", 1);
		init_signal_spy("/TB_ejecucion_normal/uut/i_ID/i_pID/i_RegisterBank/regs(3)","/TB_ejecucion_normal/spy_R3", 1);
\end{lstlisting}

\newpage

	\lstset{language=VHDL, breaklines=true, basicstyle=\footnotesize, numbers=left, firstnumber=last}
\begin{lstlisting}[frame=single]
		init_signal_spy("/TB_ejecucion_normal/uut/i_ID/i_pID/i_RegisterBank/regs(4)","/TB_ejecucion_normal/spy_R4", 1);
		init_signal_spy("/TB_ejecucion_normal/uut/i_ID/i_pID/i_RegisterBank/regs(5)","/TB_ejecucion_normal/spy_R5", 1);

		init_signal_spy("/TB_ejecucion_normal/uut/i_MEM/i_pMEM/i_MemData/mem(0)","/TB_ejecucion_normal/spy_M0",1);
		init_signal_spy("/TB_ejecucion_normal/uut/i_MEM/i_pMEM/i_MemData/mem(5)","/TB_ejecucion_normal/spy_M5",1);

		init_signal_spy("/TB_ejecucion_normal/uut/IF_out_pc_reg","/TB_ejecucion_normal/spy_PC",1);
		init_signal_spy("/TB_ejecucion_normal/uut/IF_out_inst_reg","/TB_ejecucion_normal/spy_INSTR",1);
	wait;
	end process spy_init;

	-- Test de funcionamiento
	spy_proc: process
	begin		
		-- hold reset state for 100 ns.
		rst <= '0';
		wait for 100 ns;	
		rst <= '1';

		assert spy_PC = std_logic_vector(to_unsigned(0, 32)) report "ERROR: Inicio" severity ERROR;      
	wait for clk_period*5; -- Espera para que tengan efecto los cambios (Procesador de cinco ciclos)
--LDR R2, R0, #5
		assert spy_PC = std_logic_vector(to_unsigned(20, 32)) report "ERROR: PC 20" severity ERROR;
		assert spy_R2 = std_logic_vector(to_unsigned(5, 32)) report "ERROR: LDR R2, R0, #5" severity ERROR;
	wait for clk_period;
--MOV R1, #25
		assert spy_PC = std_logic_vector(to_unsigned(24, 32)) report "ERROR: PC 24" severity ERROR;
		assert spy_R1 = std_logic_vector(to_unsigned(25, 32)) report "ERROR: MOV R1, #25" severity ERROR;
	wait for clk_period;
--MOV R3, #0
		assert spy_PC = std_logic_vector(to_unsigned(28, 32)) report "ERROR: PC 28" severity ERROR;
		assert spy_R3 = std_logic_vector(to_unsigned(0, 32)) report "ERROR: MOV R3, #0" severity ERROR;
	wait for clk_period;
--MOV R5, #1
		assert spy_PC = std_logic_vector(to_unsigned(32, 32)) report "ERROR: PC 32" severity ERROR;
\end{lstlisting}

\newpage

	\lstset{language=VHDL, breaklines=true, basicstyle=\footnotesize, numbers=left, firstnumber=last}
\begin{lstlisting}[frame=single]		
		assert spy_R5 = std_logic_vector(to_unsigned(1, 32)) report "ERROR: MOV R5, #1" severity ERROR;
	wait for clk_period;
--MOV R4, R2
		assert spy_PC = std_logic_vector(to_unsigned(36, 32)) report "ERROR: PC 36" severity ERROR;
		assert spy_R4 = std_logic_vector(to_unsigned(5, 32)) report "ERROR: MOV R4, R2" severity ERROR;
--NOP 
--NOP 
--NOP 
--CMP R4, R0 
--BEQ # 24 
--NOP 
--NOP 
--NOP
	wait for clk_period*10;
--ADD R3, R3, R1
		assert spy_PC = std_logic_vector(to_unsigned(32, 32)) report "ERROR: PC 72" severity ERROR;
		assert spy_R3 = std_logic_vector(to_unsigned(25, 32)) report "ERROR: ADD R3, R3, R1(25)" severity ERROR;
	wait for clk_period;
--SUB R4, R4, R5
		assert spy_PC = std_logic_vector(to_unsigned(36, 32)) report "ERROR: PC 76" severity ERROR;
		assert spy_R4 = std_logic_vector(to_unsigned(4, 32)) report "ERROR: SUB R4, R4, R5(4)" severity ERROR;
--CMP R4, R0 
--BEQ # 24 
--NOP 
--NOP 
--NOP
	wait for clk_period*10;
--ADD R3, R3, R1
		assert spy_PC = std_logic_vector(to_unsigned(32, 32)) report "ERROR: PC 116" severity ERROR;
		assert spy_R3 = std_logic_vector(to_unsigned(50, 32)) report "ERROR: ADD R3, R3, R1(50)" severity ERROR;
	wait for clk_period;
--SUB R4, R4, R5
		assert spy_PC = std_logic_vector(to_unsigned(36, 32)) report "ERROR: PC 120" severity ERROR;
		assert spy_R4 = std_logic_vector(to_unsigned(3, 32)) report "ERROR: SUB R4, R4, R5(3)" severity ERROR;
--CMP R4, R0 
--BEQ # 24 
--NOP 
--NOP 
--NOP
	wait for clk_period*10;
\end{lstlisting}

\newpage

	\lstset{language=VHDL, breaklines=true, basicstyle=\footnotesize, numbers=left, firstnumber=last}
\begin{lstlisting}[frame=single]
--ADD R3, R3, R1
		assert spy_PC = std_logic_vector(to_unsigned(32, 32)) report "ERROR: PC 160" severity ERROR;
		assert spy_R3 = std_logic_vector(to_unsigned(75, 32)) report "ERROR: ADD R3, R3, R1(75)" severity ERROR;
	wait for clk_period;
--SUB R4, R4, R5
		assert spy_PC = std_logic_vector(to_unsigned(36, 32)) report "ERROR: PC 164" severity ERROR;
		assert spy_R4 = std_logic_vector(to_unsigned(2, 32)) report "ERROR: SUB R4, R4, R5(2)" severity ERROR;
--CMP R4, R0 
--BEQ # 24 
--NOP 
--NOP 
--NOP
	wait for clk_period*10;
--ADD R3, R3, R1
		assert spy_PC = std_logic_vector(to_unsigned(32, 32)) report "ERROR:PC 204" severity ERROR;
		assert spy_R3 = std_logic_vector(to_unsigned(100, 32)) report "ERROR: ADD R3, R3, R1(100)" severity ERROR;
	wait for clk_period;
--SUB R4, R4, R5
		assert spy_PC = std_logic_vector(to_unsigned(36, 32)) report "ERROR: PC 208" severity ERROR;
		assert spy_R4 = std_logic_vector(to_unsigned(1, 32)) report "ERROR: SUB R4, R4, R5(1)" severity ERROR;
	wait for clk_period;
--CMP R4, R0 
--BEQ # 24 
--NOP 
--NOP 
--NOP
	wait for clk_period*9;
--ADD R3, R3, R1
		assert spy_PC = std_logic_vector(to_unsigned(32, 32)) report "ERROR: PC 248" severity ERROR;
		assert spy_R3 = std_logic_vector(to_unsigned(125, 32)) report "ERROR: ADD R3, R3, R1(125)" severity ERROR;
	wait for clk_period;
--SUB R4, R4, R5
		assert spy_PC = std_logic_vector(to_unsigned(36, 32)) report "ERROR: PC 252" severity ERROR;
		assert spy_R4 = std_logic_vector(to_unsigned(0, 32)) report "ERROR: SUB R4, R4, R5(0)" severity ERROR;
--B # -32
--NOP
--NOP
--NOP
--CMP R4, R0 
--BEQ # 24 
\end{lstlisting}

\newpage

	\lstset{language=VHDL, breaklines=true, basicstyle=\footnotesize, numbers=left, firstnumber=last}
\begin{lstlisting}[frame=single]
--NOP
--NOP
--NOP
--STR R3, R0, #0
	wait for clk_period*12;
		assert spy_PC = std_logic_vector(to_unsigned(96, 32)) report "ERROR: PC 296" severity ERROR;
		assert spy_M0 = std_logic_vector(to_unsigned(125, 32)) report "ERROR: STR R3, R0, #0" severity ERROR;
	wait;
	end process;
END;
\end{lstlisting}

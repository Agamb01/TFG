%---------------------------------------------------------------------
%
%                          config.tex
%
%---------------------------------------------------------------------
%
% Contiene la  definici�n de constantes  que determinan el modo  en el
% que se compilar� el documento.
%
%---------------------------------------------------------------------
%
% En concreto, podemos  indicar si queremos "modo release",  en el que
% no  aparecer�n  los  comentarios  (creados  mediante  \com{Texto}  o
% \comp{Texto}) ni los "por  hacer" (creados mediante \todo{Texto}), y
% s� aparecer�n los �ndices. El modo "debug" (o mejor dicho en modo no
% "release" muestra los �ndices  (construirlos lleva tiempo y son poco
% �tiles  salvo  para   la  versi�n  final),  pero  s�   el  resto  de
% anotaciones.
%
% Si se compila con LaTeX (no  con pdflatex) en modo Debug, tambi�n se
% muestran en una esquina de cada p�gina las entradas (en el �ndice de
% palabras) que referencian  a dicha p�gina (consulta TeXiS_pream.tex,
% en la parte referente a show).
%
% El soporte para  el �ndice de palabras en  TeXiS es embrionario, por
% lo  que no  asumas que  esto funcionar�  correctamente.  Consulta la
% documentaci�n al respecto en TeXiS_pream.tex.
%
%
% Tambi�n  aqu� configuramos  si queremos  o  no que  se incluyan  los
% acr�nimos  en el  documento final  en la  versi�n release.  Para eso
% define (o no) la constante \acronimosEnRelease.
%
% Utilizando \compilaCapitulo{nombre}  podemos tambi�n especificar qu�
% cap�tulo(s) queremos que se compilen. Si no se pone nada, se compila
% el documento  completo.  Si se pone, por  ejemplo, 01Introduccion se
% compilar� �nicamente el fichero Capitulos/01Introduccion.tex
%
% Para compilar varios  cap�tulos, se separan sus nombres  con comas y
% no se ponen espacios de separaci�n.
%
% En realidad  la macro \compilaCapitulo  est� definida en  el fichero
% principal tesis.tex.
%
%---------------------------------------------------------------------


% Comentar la l�nea si no se compila en modo release.
% TeXiS har� el resto.
% ���Si cambias esto, haz un make clean antes de recompilar!!!
%\def\release{1}


% Descomentar la linea si se quieren incluir los
% acr�nimos en modo release (en modo debug
% no se incluir�n nunca).
% ���Si cambias esto, haz un make clean antes de recompilar!!!
%\def\acronimosEnRelease{1}


% Descomentar la l�nea para establecer el cap�tulo que queremos
% compilar

%\compilaCapitulo{01Introduccion} % Revisado (1)
%\compilaCapitulo{02TemaFallos}   % Revisado (1)
\compilaCapitulo{03TrabajoPropio}
%\compilaCapitulo{05Resultados}
%\compilaCapitulo{06AnalisisResultados}
%\compilaCapitulo{07Conclusiones}
%\compilaCapitulo{ParaCorreccion}
%\compilaCapitulo{Trabajando}

% \compilaApendice{01AsiSeHizo}

% Variable local para emacs, para  que encuentre el fichero maestro de
% compilaci�n y funcionen mejor algunas teclas r�pidas de AucTeX
%%%
%%% Local Variables:
%%% mode: latex
%%% TeX-master: "./Tesis.tex"
%%% End:

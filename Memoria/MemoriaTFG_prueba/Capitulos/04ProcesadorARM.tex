%---------------------------------------------------------------------
%
%                          Cap�tulo 4
%
%---------------------------------------------------------------------

\chapter{Procesador ARM}
\label{cap4}

\begin{FraseCelebre}
\begin{Frase}
...
\end{Frase}
\begin{Fuente}
...
\end{Fuente}
\end{FraseCelebre}

\begin{resumen}
...
\end{resumen}


%-------------------------------------------------------------------
\section*{Introducci�n}
%-------------------------------------------------------------------
\label{cap4:sec:introduccion}

...

%-------------------------------------------------------------------
\section{Arquitectura del Cortex-M4}
%-------------------------------------------------------------------
\label{cap4:sec:cortexm4}

...

%-------------------------------------------------------------------
\section{Nuestra Arquitectura}
%-------------------------------------------------------------------
\label{cap4:sec:nuestra}


asdsadsadsa d Ss dsad sa�


sad sad sad 
...

%%-------------------------------------------------------------------
%\section*{\NotasBibliograficas}
%%-------------------------------------------------------------------
%\TocNotasBibliograficas
%
%Citamos algo para que aparezca en la bibliograf�a\ldots
%\citep{ldesc2e}
%
%\medskip
%
%Y tambi�n ponemos el acr�nimo \ac{CVS} para que no cruja.
%
%Ten en cuenta que si no quieres acr�nimos (o no quieres que te falle la compilaci�n en ``release'' mientras no tengas ninguno) basta con que no definas la constante \verb+\acronimosEnRelease+ (en \texttt{config.tex}).
%
%
%%-------------------------------------------------------------------
%\section*{\ProximoCapitulo}
%%-------------------------------------------------------------------
%\TocProximoCapitulo
%
%...

% Variable local para emacs, para  que encuentre el fichero maestro de
% compilaci�n y funcionen mejor algunas teclas r�pidas de AucTeX
%%%
%%% Local Variables:
%%% mode: latex
%%% TeX-master: "../Tesis.tex"
%%% End:

%---------------------------------------------------------------------
%
%                          Cap�tulo 2
%
%---------------------------------------------------------------------

\chapter{Titulo por definir}
\label{cap2}

\begin{FraseCelebre}
  \begin{Frase}
    "`La verdadera ciencia ense�a, sobre todo, a dudar y a ser ignorante."' 
  \end{Frase}
  \begin{Fuente}
    Ernest Rutherford 
  \end{Fuente}
\end{FraseCelebre}

\begin{resumen}
En este cap�tulo se define con detalle lo que es un procesador y su importancia en el mundo hoy en d�a. Tambi�n se habla sobre una arquitectura m�s concreta, la arquitectura ARM.
A continuaci�n se define qu� es un fallo y qu� tipos de fallos pueden ocurrir en los sistemas. Adem�s se explican algunas t�cnicas de tolerancia a fallos.
Para terminar se justifica la importancia de la tolerancia en los sistemas y concretamente porque es necesaria la tolerancia en los microprocesadores.
\end{resumen}


%-------------------------------------------------------------------
\section*{Introducci�n}
%-------------------------------------------------------------------
\label{cap2:introduccion}

...


%-------------------------------------------------------------------
\section{Procesador}
%-------------------------------------------------------------------
\label{cap2:procesador}

El diccionario de la Real Academia Espa�ola define al procesador como la "`unidad central de proceso (CPU), formada por uno o dos chips"'. La CPU es el circuito integrado encargado de acceder a las instrucciones de los programas inform�ticos, analizarlas y ejecutarlas realizando operaciones aritm�ticas, l�gicas y de entrada/salida. 

Von neumann

Tareas del procesador:

\begin{enumerate}
  \item Acceder a las instrucciones almacenadas en memoria.
  \item Analizar las instrucciones y establecer las se�ales de control internas.
  \item Ejecutar operaciones sobre datos.
  \item Almacenar los resultados en memoria.
\end{enumerate}

% Bibliograf�a 

%-------------------------------------------------------------------
\subsection*{Instrucciones}
%-------------------------------------------------------------------
\label{cap2:procesador:instrucciones}


%-------------------------------------------------------------------
\subsection*{Instrucciones}
%-------------------------------------------------------------------
\label{cap2:procesador:instrucciones}


%-------------------------------------------------------------------
\subsection*{Memoria}
%-------------------------------------------------------------------
\label{cap2:procesador:memoria}

%-------------------------------------------------------------------
\subsection*{Direccionamiento}
%-------------------------------------------------------------------
\label{cap2:procesador:direccionamiento}

%-------------------------------------------------------------------
\subsection*{Segmentaci�n}
%-------------------------------------------------------------------
\label{cap2:procesador:segmentacion}


Memoria 

Entrada/salida


%-------------------------------------------------------------------
\subsection{DLX}
%-------------------------------------------------------------------
\label{cap2:procesador:dlx}

...


%-------------------------------------------------------------------
\subsection{ARM}
%-------------------------------------------------------------------
\label{cap2:procesador:arm}

...

%-------------------------------------------------------------------
\section{Fallos}
%-------------------------------------------------------------------
\label{cap2:fallos}

Existen una gran variedad de fallos que pueden ocurrir en un sistema electr�nico. Los fallos se pueden clasificar en fallos software y fallos hardware. Y los podemos encontrar desde fallos en la definici�n de requisitos que se propagan hasta la fase de producci�n, hasta fallos producidos en el sistema por agentes externos como la radiaci�n. 

En esta secci�n se hablar� de esta �ltima clase de fallos, los fallos producidos por agentes externos que no se pueden evitar en las fases de dise�o. Y que afectan al hardware, da�ando sus componentes o alterando los valores de las se�ales con las que se trabaja.


Los fallos analizados en esta secci�n se van a dividir en dos categor�as; fallos permanentes y fallos transitorios.

%-------------------------------------------------------------------
\subsection{Fallos Permanentes}
%-------------------------------------------------------------------
\label{cap2:fallos:fallospermanentes}

Los fallos permanentes son aquellos que afectan al sistema de forma irreversible. Producen cambios en el dise�o que estropean el correcto funcionamiento del m�dulo o circuito que lo sufre. Estos fallos no se solucionan reiniciando el sistema.

%bibliografia: jesd89a.pdf

critical charge:

hard error:


single-event functional interrupt (SEFI):

single-event latch-up (SEL):

single event transient (SET):

single-event upset (SEU):

single-event upset (SEU) rate:

soft error, device:

soft error rate (SER):





%-------------------------------------------------------------------
\subsection{Fallos Transitorios}
%-------------------------------------------------------------------
\label{cap2:fallos:fallostransitorios}

...



%-------------------------------------------------------------------
\section{Tolerancia a Fallos}
%-------------------------------------------------------------------
\label{cap2:tolerancia}

la tolerancia a fallos se define como la capacidad de un sistema de funcionar correctamente 

Existen dos tipos de tolerancia; tolerancia est�tica y tolerancia din�mica.

%-------------------------------------------------------------------
\subsection{Tolerancia est�tica}
%-------------------------------------------------------------------
\label{cap2:tolerancia:toleranciaestatica}

...

%-------------------------------------------------------------------
\subsection{Tolerancia en din�mica}
%-------------------------------------------------------------------
\label{cap2:tolerancia:toleranciadinamica}

...

%-------------------------------------------------------------------
\subsection{Tolerancia en microprocesadores}
%-------------------------------------------------------------------
\label{cap2:tolerancia:microprocesadores}

...


%%-------------------------------------------------------------------
%\section*{\NotasBibliograficas}
%%-------------------------------------------------------------------
%\TocNotasBibliograficas
%
%Citamos algo para que aparezca en la bibliograf�a\ldots
%\citep{ldesc2e}
%
%\medskip
%
%Y tambi�n ponemos el acr�nimo \ac{CVS} para que no cruja.
%
%Ten en cuenta que si no quieres acr�nimos (o no quieres que te falle la compilaci�n en ``release'' mientras no tengas ninguno) basta con que no definas la constante \verb+\acronimosEnRelease+ (en \texttt{config.tex}).
%
%
%%-------------------------------------------------------------------
%\section*{\ProximoCapitulo}
%%-------------------------------------------------------------------
%\TocProximoCapitulo
%
%...

% Variable local para emacs, para  que encuentre el fichero maestro de
% compilaci�n y funcionen mejor algunas teclas r�pidas de AucTeX
%%%
%%% Local Variables:
%%% mode: latex
%%% TeX-master: "../Tesis.tex"
%%% End:

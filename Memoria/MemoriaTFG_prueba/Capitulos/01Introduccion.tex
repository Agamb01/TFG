%---------------------------------------------------------------------
%
%                          Cap�tulo 1
%
%---------------------------------------------------------------------

\chapter{Introducci�n}
\label{cap1}

\begin{FraseCelebre}
\begin{Frase}
...
\end{Frase}
\begin{Fuente}
...
\end{Fuente}
\end{FraseCelebre}

\begin{resumen}
En el cap�tulo 1 se realiza una introducci�n al trabajo realizado durante el proyecto. Se plantea el problema, se enumeran los objetivos del trabajo y se define la estructura de este documento.
\end{resumen}

%%-------------------------------------------------------------------
\section{Introducci�n}
%%-------------------------------------------------------------------
\label{cap1:introduccion}

Esta memoria es el resultado del trabajo realizado para la asignatura "`Trabajo de fin de grado"' del \grado. Trabajo realizado en el departamento de Arquitectura de Computadores y Autom�tica con \director como director.

Este trabajo se centra en la implementaci�n de un microprocesador tolerante a fallos. El procesador ha sido dise�ado para ser compatible con instrucciones ARM, del repertorio del microprocesador "`ARM Cortex M3"' [\cite{Sadasivan2006}]. Y la tolerancia a fallos aplicada ha sido el "`modelo de replicado triple de m�dulos(TMR)"' [\cite{Habinc2002}].



%%-------------------------------------------------------------------
\section{Motivaci�n}
%%-------------------------------------------------------------------
\label{cap1:motivacion}

%%-------------------------------------------------------------------
\section{Antecedentes}
%%-------------------------------------------------------------------
\label{cap1:antecedentes}

%%-------------------------------------------------------------------
\section{Planteamiento del problema}
%%-------------------------------------------------------------------
\label{cap1:planteamiento}

Este trabajo tiene en cuenta que el motor principal de muchos sistemas es el microprocesador. El procesador de un sistema es su cerebro, m�s concretamente es el encargado de ejecutar las instrucciones que componen los programas. 

Si no se toman medidas de prevenci�n y tolerancia, este cerebro puede ver alterado su comportamiento por efectos externos, provocando errores de ejecuci�n. Estos errores a su vez pueden pueden ser causa de un comportamiento no deseado modificando el funcionamiento del propio procesador o de otros componentes del sistema (memorias, controladores entrada/salida).

Con este trabajo se quiere evitar estas situaciones concediendo un grado extra de fiabilidad a los sistemas basados en micro-procesado. Para ello se quiere dise�ar e implementar un microprocesador sencillo capaz de ejecutar un conjunto reducido de instrucciones (RISC). Al que posteriormente se le aplicar�n t�cnicas de tolerancia a fallos. Aumentando as� su capacidad de detectar e incluso recuperarse de los fallos. 


%%-------------------------------------------------------------------
\section{Objetivos}
%%-------------------------------------------------------------------
\label{cap1:objetivos}

El proyecto se ha divido en cuatro tareas dedicadas a la implementaci�n del microprocesador y a la aplicaci�n de la tolerancia a fallos.

\begin{enumerate}
  \item Primero se ha implementado el procesador segmentado en 5 etapas. Para ello se ha partido de la arquitectura DLX vista en las asignaturas de computadores de nuestro grado. 
	\item A continuaci�n se ha redise�ado la ruta de control. El nuevo juego de instrucciones usado, completamente distinto al que utiliza un DLX convencional, obliga a cambiar la ruta de control. Simulando unos peque�os programas se comprueba que el procesador es capaz de decodificar y ejecutar las nuevas instrucciones. 
	\item Una vez implementado el procesador completo y comprobado su funcionamiento se dise�a y se incorpora la tolerancia a fallos. Para ello se triplican los m�dulos que pueden causar mayor numero de fallos y se insertan votadores de mayor�a. %Estos votadores reciben la salida de un m�dulo y sus dos replicas y permiten conocer el valor de la mayor�a.
	\item Para finalizar se ha dise�ado un sistema externo de inserci�n de fallos. Este sistema es capaz de alterar los valores de las salidas de los m�dulos triplicados, para comprobar despu�s como afecta esto al funcionamiento del procesador.
\end{enumerate}


%%-------------------------------------------------------------------
\section{Estructura del documento}
%%-------------------------------------------------------------------
\label{cap1:estructura}

A continuaci�n se presenta c�mo se ha estructurado el contenido de esta memoria.

\begin{description}
  \item Capitulo \ref{cap1} \textit{Introducci�n}: 
	
	En el capitulo 1 se realiza la introducci�n al trabajo propuesto y realizado para el trabajo de fin de grado en el que se basa el presente documento.
	
  \item Capitulo \ref{cap2} \textit{Trabajo relacionado}:
	
	En el capitulo 2 se realiza una exposici�n de la investigaci�n llevada a cabo para realizar el trabajo. Se exponen definiciones y otros datos de inter�s para el lector.

	\item Capitulo \ref{cap3} \textit{Procesador}: 
	
	En el capitulo 3 se describe el procesador implementado con su estructura y arquitectura.
	
	\item Capitulo \ref{cap4} \textit{Proporcionando tolerancia a fallos}: 
	
	En el capitulo 4 se describe c�mo se ha proporcionado la tolerancia a fallos y qu� t�cnicas se han utilizado.
	
	\item Capitulo \ref{cap5} \textit{Resultados}: 
	
	En el capitulo 5 se muestran los resultados obtenidos de las simulaciones realizadas.
	
	\item Capitulo \ref{cap6} \textit{An�lisis de los resultados}: 
	
	En el capitulo 6 se analizan los datos.
	
	\item Capitulo \ref{cap7} \textit{Conclusiones}: 
	
	En el capitulo 7 se describen las conclusiones tras analizar los resultados.
	
\end{description}





%%-------------------------------------------------------------------
%\section*{\NotasBibliograficas}
%%-------------------------------------------------------------------
%\TocNotasBibliograficas
%
%Citamos algo para que aparezca en la bibliograf�a\ldots
%\citep{ldesc2e}
%
%\medskip
%
%Y tambi�n ponemos el acr�nimo \ac{CVS} para que no cruja.
%
%Ten en cuenta que si no quieres acr�nimos (o no quieres que te falle la compilaci�n en ``release'' mientras no tengas ninguno) basta con que no definas la constante \verb+\acronimosEnRelease+ (en \texttt{config.tex}).
%
%
%%-------------------------------------------------------------------
%\section*{\ProximoCapitulo}
%%-------------------------------------------------------------------
%\TocProximoCapitulo
%
%...

% Variable local para emacs, para  que encuentre el fichero maestro de
% compilaci�n y funcionen mejor algunas teclas r�pidas de AucTeX
%%%
%%% Local Variables:
%%% mode: latex
%%% TeX-master: "../Tesis.tex"
%%% End:

%---------------------------------------------------------------------
%
%                          Cap�tulo 7
%
%---------------------------------------------------------------------

\chapter{Conclusions} % y futuras l�neas de trabajo}
\label{conclusions}

%\begin{FraseCelebre}
%\begin{Frase}
%...
%\end{Frase}
%\begin{Fuente}
%...
%\end{Fuente}
%\end{FraseCelebre}

\section{Conclusions}
\label{conclusions:conclusions}

...

%\section{Futuras l�neas de trabajo}
%\label{conclusions:futuras}

%Las posibilidades futuras lineas de trabajo para continuar con este desarrollo de este proyecto podr�an incluir ampliaciones en la funcionalidad del sistema o aumentar la tolerancia a fallos. 
%
%Si se desea ampliar la funcionalidad se podr� conseguir aumentando el repertorio de instrucciones. En nuestro caso solo se han implementado instrucciones que permiten ejecutar programas b�sicos. Igualmente se pueden modificar las memorias de datos e instrucciones para permitir el uso de memorias externas e insertando memorias cach�. O bien se puede integrar la CPU junto a controladores de diferentes tipos.
%
%Por otro lado, si se desea aumentar la tolerancia a fallos, se puede aplicar la t�cnica TMR sobre otros componentes de la misma forma que se han aplicado sobre los registros de control. Tambi�n se pueden aplicar nuevas t�cnicas para tolerar otros tipos de fallos, tal es el caso de aplicar la re-configuraci�n est�tica o din�mica para que el sistema sea capaz de tolerar fallos permanentes adem�s de los fallos transitorios.







%---------------------------------------------------------------------
%
%                          Cap�tulo 7
%
%---------------------------------------------------------------------

\chapter{Conclusiones y futuras l�neas de trabajo}
\label{conclusiones}

%\begin{FraseCelebre}
%\begin{Frase}
%...
%\end{Frase}
%\begin{Fuente}
%...
%\end{Fuente}
%\end{FraseCelebre}

\section{Conclusiones}

El objetivo de este proyecto era crear una CPU segmentada compatible con ARM y dotarla de tolerancia a fallos transitorios con la intenci�n de conseguir una ejecuci�n correcta en sistemas donde los fallos fuesen algo com�n, como en sistemas internos de sat�lites o aeronaves. 

En primera instancia se pretend�a compatibilizar la CPU con un repertorio de instrucciones ampliamente usado. Se decidi� utilizar el repertorio ARM ya que forma parte de una de las arquitecturas m�s utilizadas en todo tipo de sistemas. Para conseguir ejecutar estas instrucciones hubo que implementar una nueva unidad de control capaz de descifrar las instrucciones e integrarla en la CPU. 

Una vez se tuvo el procesador con un repertorio de instrucciones capaz de ejecutar programas sencillos, se quer�a asegurar que este funcionar�a bajo condiciones donde los fallos aparecieran con una alta probabilidad. Se decidi� que los registros internos de la CPU eran los elementos que provocar�an un mayor impacto, estos son muy numerosos y cada ciclo de reloj almacenan nueva informaci�n, si uno fallase se propagar�a alterando la instrucci�n y el programa que se estuviera ejecutando.

Se decidi� aplicar la t�cnica de "`Triple Modular Redundancy"' por ser una t�cnica efectiva y ampliamente utilizada en sistemas que requieren un mayor grado de tolerancia frente a fallos transitorios.

Una vez implementado el dise�o y a�adida la tolerancia a fallos realizaron simulaciones de situaciones controladas para comprobar si esta t�cnica realmente permit�a al procesador funcionar bajo las condiciones deseadas.

Se llevaron a cabo simulaciones funcionales donde se ejecutaba un programa sobre ambas CPUs, el programa se ejecutaba 

Se llevaron a cabo simulaciones funcionales sobre ambas CPUs. En unas simulaciones se ejecutaba un programa sin interferencias, mientras que en otras se ejecutaba el mismo programa insertando fallos en algunos registros. Con estas simulaciones pudimos determinar que:

\begin{enumerate}
  \item Los resultados de la prueba de control eran id�nticos por lo tanto ambas CPUs ejecutaban correctamente las instrucciones del programa en un entorno sin fallos.
	\item Al insertar los mismos fallos en ambas CPUs, se observaban resultados diferentes. Comparando los resultados de la CPU est�ndar con la prueba anterior se comprob� que los fallos hab�an provocado errores graves en la ejecuci�n del programa. Mientras que insertando los mismos fallos en la CPU tolerante a fallos daba los mismos resultados que las pruebas de control.
\end{enumerate}

Gracias a estas simulaciones se pudo demostrar que se ha dise�ado una unidad de procesamiento capaz de ejecutar programas sencillos y de tolerar fallos transitorios individuales de manera satisfactoria.


\section{Futuras l�neas de trabajo}

Algunas posibilidades para continuar con este trabajo podr�an incluir ampliaciones en la funcionalidad del sistema o aumentar la tolerancia a fallos. 

Si deseamos ampliar la funcionalidad se puede conseguir aumentando el repertorio de instrucciones ya que solo se han implementado una pocas instrucciones que permiten ejecutar programas b�sicos en el dise�o actual. Modificar las memorias de datos e instrucciones para permitir el uso de memorias externas e insertando memorias cach�. O bien se puede integrar la CPU junto a controladores de diferentes tipos.

Por otro lado, si se desea aumentar la tolerancia a fallos, se puede aplicar la t�cnica TMR sobre otros componentes de la misma forma que se han aplicado sobre los registros de control. O se pueden aplicar nuevas t�cnicas para tolerar otros tipos de fallos, como podr�a aplicarse la re-configuraci�n est�tica o din�mica para que el sistema fuese capaz de tolerar fallos permanentes adem�s de los fallos transitorios.







%---------------------------------------------------------------------
%
%                          Cap�tulo 8
%
%---------------------------------------------------------------------

\chapter{Conclusiones} % y futuras l�neas de trabajo}
\label{conclusiones}

%\begin{FraseCelebre}
%\begin{Frase}
%...
%\end{Frase}
%\begin{Fuente}
%...
%\end{Fuente}
%\end{FraseCelebre}

\section{Conclusiones}

Se ha dise�ado una Unidad Central de Proceso (CPU) segmentada en 5 etapas, capaz de ejecutar un conjunto reducido de instrucciones conocido como THUMB-2, concretamente es un subconjunto de la arquitectura ARM. Esta CPU ejecuta correctamente instrucciones aritm�tico-l�gicas, instrucciones de acceso a memoria e instrucciones de bifurcaci�n o salto.

Una vez obtenido el procesador con un repertorio de instrucciones capaz de ejecutar programas sencillos, se ha aumentado el grado de fiabilidad de la CPU dise�ada aplicando la t�cnica de tolerancia a fallos "`Triple Modular Redundancy"' (TMR). Se han triplicado los registros que almacenan informaci�n entre las etapas de la CPU segmentada e insertado votadores de mayor�a para obtener los valores correctos.

Para finalizar se ha demostrado, a base de simulaciones sobre el procesador dise�ado en este proyecto, que al aplicar t�cnicas como la TMR en la CPU se han conseguido tolerar fallos que habr�an alterado el funcionamiento del programa. El mecanismo de tolerancia aplicado ha permitido tolerar un fallo en cada elemento triplicado.



%El objetivo de este proyecto es crear una CPU segmentada compatible con ARM y dotarla de tolerancia a fallos transitorios con la intenci�n de conseguir una ejecuci�n correcta en aquellos sistemas susceptibles de sufrir, caso de los en sistemas internos de sat�lites o de aeronaves. 

%Una vez obtenido el procesador con un repertorio de instrucciones capaz de ejecutar programas sencillos, se ha de asegurar que �ste funciona bajo condiciones donde los fallos aparecen con una alta probabilidad. Se decide que los registros internos de la CPU son los elementos que provocan un mayor impacto, ya que estos son muy numerosos y cada ciclo de reloj almacenan nueva informaci�n, si uno fallase se propagar�a alterando la instrucci�n y el programa que se estuviera ejecutando.

%Se aplica la t�cnica de "`Triple Modular Redundancy"' por ser una t�cnica efectiva y ampliamente utilizada en sistemas que requieren un mayor grado de tolerancia frente a fallos transitorios.

%Una vez implementado el dise�o y a�adida la tolerancia a fallos se realizan las simulaciones de situaciones controladas para comprobar si esta t�cnica realmente permite que el procesador funcione bajo las condiciones deseadas.

%\newpage
%
%Con este objetivo se llevan a cabo simulaciones funcionales consistentes en ejecutar un programa sobre ambas CPUs.
%
%En unas simulaciones se ejecuta un programa sin interferencias, mientras que en otras se ejecuta el mismo programa insertando fallos en algunos registros. Con estas simulaciones se ha determinado que:
%
%\begin{enumerate}
  %\item Los resultados de la prueba de control son id�nticos, por lo tanto ambas CPUs ejecutan correctamente las instrucciones del programa en un entorno sin fallos.
	%
	%\item Al insertar los mismos fallos en ambas CPUs, se han observado resultados diferentes. Comparando los resultados de la CPU est�ndar con la prueba anterior se comprueba que los fallos provocan errores graves en la ejecuci�n del programa. Mientras que insertando los mismos fallos en la CPU tolerante a fallos se obtienen los mismos resultados que en las pruebas de control.
%\end{enumerate}
%
%Gracias a estas simulaciones se ha podido demostrar que el dise�o de la Unidad Central de Procesamiento es capaz de ejecutar programas y de tolerar fallos transitorios individuales de manera satisfactoria.
%
%
%\section{Futuras l�neas de trabajo}
%
%Las posibilidades futuras lineas de trabajo para continuar con este desarrollo de este proyecto podr�an incluir ampliaciones en la funcionalidad del sistema o aumentar la tolerancia a fallos. 
%
%Si se desea ampliar la funcionalidad se podr� conseguir aumentando el repertorio de instrucciones. En nuestro caso solo se han implementado instrucciones que permiten ejecutar programas b�sicos. Igualmente se pueden modificar las memorias de datos e instrucciones para permitir el uso de memorias externas e insertando memorias cach�. O bien se puede integrar la CPU junto a controladores de diferentes tipos.
%
%Por otro lado, si se desea aumentar la tolerancia a fallos, se puede aplicar la t�cnica TMR sobre otros componentes de la misma forma que se han aplicado sobre los registros de control. Tambi�n se pueden aplicar nuevas t�cnicas para tolerar otros tipos de fallos, tal es el caso de aplicar la re-configuraci�n est�tica o din�mica para que el sistema sea capaz de tolerar fallos permanentes adem�s de los fallos transitorios.







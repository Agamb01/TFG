%-----------------------------------------------------------------
\subsubsection{Segmentaci�n DLX}
%-----------------------------------------------------------------
\label{cap2:procesador:dlx:segmentacion}

El DLX basa su rendimiento en la segmentaci�n que se divide en 5 etapas

\begin{itemize}
	\item \textit{B�squeda de la instrucci�n (IF)} 
	
	Esta primera etapa es la encargada de acceder a memoria y traer la siguiente instrucci�n.
	
	\item \textit{Descodificaci�n de la instrucci�n (ID)}
	
	En la segunda etapa se descodifica la instrucci�n cargada en la primera etapa, obteniendo las se�ales de control. Extrae los operandos del banco de registro o de la propia instrucci�n.
	
	\item \textit{Ejecuci�n y c�lculo de direcciones efectivas (EX)}
	
	La tercera etapa se encarga de ejecutar la instrucci�n utilizando las unidades funcionales. Las unidades funcionales pueden estar segmentadas y/o duplicadas.
	
	\item \textit{Acceso a memoria (MEM)}
	
	En la etapa de memoria es cuando se ejecutan las operaciones de carga y almacenamiento. Las instrucciones de carga traen datos de la memoria y los almacenan en los registros, mientras que las instrucciones de almacenamiento guardan los datos en memoria.
	
	\item \textit{Postescritura (WB)}
	
	En la �ltima etapa se almacenan los resultados de las instrucciones en los registros.
	
\end{itemize}

Como pudimos ver en la figura \ref{tab:segmentacion}, las instrucciones se buscan en cada ciclo de reloj, a menos que surjan riesgos debido a la segmentaci�n. Como vimos en el apartado \ref{cap2:procesador:segmentacion:riesgos}, la segmentaci�n implica ciertos riesgos. Para solucionar o reducir estos, el DLX implementa las siguientes t�cnicas \cite{Hennessy2006}:

\begin{enumerate}
  \item Duplicar y/o segmentar las unidades funcionales. Con ello se reducen los ciclos de espera debidos a los \textit{riesgos estructurale}s. Se consigue un mayor n�mero de etapas para poder cargar nuevas instrucciones.
	
	\item "`Adelantamiento"'(forwarding) o "`Cortocircuito"'. T�cnica encaminada a resolver los \textit{riesgos de datos}. Se consigue proporcionar un acceso a los resultados de instrucciones previas que todav�a no han almacenado los datos en el "`banco de registros"'. 
		\begin{itemize}
		
		  \item \textsl{Lectura despu�s de escritura (RAW)}. Debido al cortocircuito implementado, el dato es recibido de las etapas siguientes y no es necesario que se haya escrito en los registros.
			
			\item \textsl{Escritura despu�s de lectura (WAR)}. No puede ocurrir debido a que todas las lecturas se realizan al comienzo de la ejecuci�n, en la etapa de descodificaci�n, y las escrituras al final, en la etapa de postescritura.
			
			\item \textsl{Escritura despu�s de escritura (WAW)}. Solo se presenta en segmentaciones que escriben en m�s de una etapa, esta arquitectura no se ve afectada ya que solo escribe en la etapa de postescritura. puede ocurrir de
			
	  \end{itemize}
		
	\item \textit{Riesgos de control}. Si se ejecuta una instrucci�n de salto, el cambio no se ve reflejado hasta la fase de memoria, esto implica una detenci�n de 3 ciclos. En DLX se utiliza l�gica especializada para averiguar si el salto es efectivo y para la calcular la direcci�n destino de salto en la etapa de descodificaci�n. 
	
\end{enumerate}

Cuando existe un riesgo que no es posible evitar con estas t�cnicas se aplica un interbloqueo de la segmentaci�n. Este recurso act�a de modo que cuando se detecta un riesgo se detiene la ejecuci�n de la instrucci�n hasta que el riesgo desaparece. El bloqueo se realiza en la fase de decodificaci�n, donde es posible determinar si existe alg�n riesgo.
